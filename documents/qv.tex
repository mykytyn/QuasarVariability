% This file is part of the QuasarVariability project
% Copyright 2013 David W. Hogg (NYU) and any other authors.

\documentclass[letterpaper,12pt]{article}

\newcommand{\project}[1]{\textsl{#1}}
\newcommand{\sdss}{\project{SDSS}}
\newcommand{\panstarrs}{\project{PanSTARRS}}
\newcommand{\given}{\,|\,}
\newcommand{\transpose}[1]{{#1}^{\mathsf{T}}}
\newcommand{\inverse}[1]{{#1}^{-1}}

\begin{document}

\section{The model}

In our language, a ``model'' is a likelihood function---a function
equal to or proportional to a pdf in data space for the data given
parameters---and a set of prior pdfs for some or all of those
parameters.  For this project, the ``data'' for one quasar comprise a
set of $N$ observed magnitudes $m_n$.  About each data point we have
various bits of (assumed correct) meta data: We know the time $t_n$ at
which each measurement was made.  We know the astronomical bandpass
$b_n$ through which the measurement was made.  The bandpass variable
$b_n$ can only take on one of a small number of possible integer
values (5 in the case of \sdss\ and \panstarrs, each of which observe
in 5 substantially distinct bandpasses).  We also know the variance
$\sigma_n^2$ of the noise contribution to the measurement $m_n$, which
we will assume is not only correct but also represents the variance of
a Gaussian pdf for the noise.  That is, we are assuming Gaussian
uncertainties of known (though heteroskedastic) variance.

To generate the likelihood function we use here, we make
use of a Gaussian Process formulation for stochastic quasar
variability; this formulation can encompass damped random walks
and also power-law structure functions (and indeed many other kinds of
continuous stochastic variability).  The unusual aspect of the model
we use here is that the Gaussian Process is not in any particular band
but instead in an ersatz fiducial band which can be scaled and shifted
onto the particular bands.  This permits simultaneous treatment of
multiple bands, even when the multiple bands are not observed
simultaneously.  The key idea is that the ersatz band is a latent
variable---it is never directly observed; only the scaled and shifted
versions are observed, and even these are only observed in the
presence of substantial measurement noise.

In the damped random walk model, without loss of generality (the real
photometric measurements will be scaled and shifted to match the
latent ersatz basis), the ersatz lightcurve can be described with a
zero-mean and unit-characteristic-variance Gaussian Process.  That is,
the prior pdf for a set of $N$ ersatz ``magnitudes'' $q_n$ that are
instantiated at times $t_n$ is just
\begin{eqnarray}
p(q) &=& N(q\given 0,V)
\\
\transpose{q} &\equiv& [q_1, q_2, \cdots , q_N]
\\
V_{nn'} &=& \exp -\frac{|t_n - t_{n'}|}{\tau}
\quad ,
\end{eqnarray}
where $N(x\given\mu,V)$ is the multivariate normal for column vector
$x$ given mean vector $\mu$ and general variance tensor $V$, $q$ is
the $N$-dimensional column vector made up of all the latent ersatz
magnitudes $q_n$, $0$ is not zero but the $N$-dimensional
generalization of zero, $V$ is a $N\times N$ symmetric positive
definite matrix with elements $V_{nn'}$, and $\tau$ is the
decorrelation time of the random walk.  Some draws from this kind of
Gaussian Process are shown in \figurename~\ref{fig:qdraws}; it is a
model of stochastic variation with controlled correlation properties.

The pdf for an individual measurement $m_n$ given its meta data and a
value for the corresponding latent ersatz magnitude $q_n$ is found by
shifting and scaling the latent magnitude and adding Gaussian noise.
This makes a single-datum likelihood
\begin{eqnarray}
p(m_n\given q_n,b_n,\sigma_n^2) &=& N(m_n\given a(b_n) * q_n + \mu(b_n), \sigma_n^2)
\quad ,
\end{eqnarray}
where $a(b)$ is a scale appropriate to photometric bandpass $b$,
$\mu(b)$ is an offset or mean appropriate to bandpass $b$, and there
is one value of $a(b)$ and one value of $\mu(b)$ for each possible
value of the bandpass $b$.  Because everything is Gaussian, the latent
ersatz magnitudes never have to be explicitly inferred individually;
they can all be marginalized out analytically.

This marginalization (to remove all explicit mention of the ersatz
variable $q$) leads to the following very simple covariance function
for the data:
\begin{eqnarray}
p(m) &=& N(m\given \mu,V)
\label{eq:likestart}
\\
\transpose{m} &\equiv& [m_1, m_2, \cdots , m_N]
\\
\transpose{\mu} &=& [\mu(b_1), \mu(b_2), \cdots , \mu(b_N)]
\\
V_{nn'} &=& a(b_n)\,a(b_{n'})\,\exp -\frac{|t_n - t_{n'}|}{\tau} + \sigma_n^2\,\delta_{nn'}
\label{eq:likeend}
\quad ,
\end{eqnarray}
where $m$ is a column vector of all the observations (from all bands),
$\mu$ is a column vector of all the means, with each mean value in the
vector drawn from the list of five means $\mu(b)$ but appropriately
for the data point in question, the rows and columns of the variance
tensor $V$ have been multiplied by the scales $a(b)$ appropriate for
the relevant observations, and there is an additional term
$\sigma_n^2$ added only to the diagonal elements to model the
measurement noise variance.  This model is very rigid, in that the
bandpasses are tightly tied to one another, but also very flexible, in
that the quasar can have any mean color and that color can change
(deterministically) as it brightens (as is observed [HOGG CITE]).
Some draws from this multi-band damped-random-walk Gaussian Process
are shown in \figurename~\ref{fig:mdraws}

The equations~(\ref{eq:likestart}) through (\ref{eq:likeend})
provide a method for computing the probability of any data set (any
set of observed magnitudes $m_n$) given meta data (times $t_n$,
bandpasses $b_n$, observational uncertainties $\sigma_n^2$) and
parameters $a(b)$ (five numbers), $\mu(b)$ (five numbers), and $\tau$.
The probability of the observed data given parameters is the
\emph{likelihood function}.  Thus we can bring the whole machinery of
probabilistic inference (likelihood optimization, posterior sampling)
to the problem.  Also, because both the process is Gaussian and the
noise is (assumed to be) Gaussian, it is trivial to make prior draws
of expected data, and also make \emph{conditional predictions} for new
data $\tilde{m}_k$ at $K$ times $t_k$ taken through bandpasses $b_k$ with
expected uncertainties $\sigma_k^2$ given the data in hand (which are
at $N$ times $t_n$):
\begin{eqnarray}
p(\tilde{m}|m) &=& N(\tilde{m}|\tilde{\mu},\tilde{V})
\\
\transpose{\tilde{m}} &\equiv& [\tilde{m}_1, \tilde{m}_2, \cdots , \tilde{m}_K]
\\
\tilde{\mu} &=& \nu + X\cdot\inverse{V}\cdot [m - \mu]
\\
\tilde{V} &=& Y - X\cdot\inverse{V}\cdot\transpose{X}
\\
\transpose{\nu} &=& [\mu(b_1), \mu(b_2), \cdots , \mu(b_K)]
\\
X_{kn} &=& a(b_k)\,a(b_n)\,\exp -\frac{|t_k - t_n|}{\tau}
\\
Y_{kk'} &=& a(b_k)\,a(b_{k'})\,\exp -\frac{|t_k - t_{k'}|}{\tau} + \sigma_k^2\,\delta_{kk'}
\quad ,
\end{eqnarray}
where $\tilde{m}$ is the column vector of conditional predictions,
$\tilde{\mu}$ and $\tilde{V}$ are a conditional mean vector and a
conditional variance tensor, (temporary) mean vector $\nu$ is
$K$-dimensional, and the matrices $V$, $X$, and $Y$ are $N\times N$,
$K\times N$, and $K\times K$ respectively.  Vectors $m$ and $\mu$ and
matrix $V$ are defined above in equations~(\ref{eq:likestart}) through
(\ref{eq:likeend}).  These conditional predictions depend on the
Gaussian Process parameters.  They represent an adjusted Gaussian
Process, adjusted to ``go through'' (or at least near) the existing
data.  We show an example of conditional predictions in
\figurename~\ref{fig:conditional}

\section{experiments}

\section{discussion}

%\acknowledgements
Thank Zoubin for the tutorial!

\clearpage
\begin{figure}
~[DM + EP]~
\caption{Four draws from the damped random walk model for the latent
  variable $q$.  These draws are made with $a=0.75$~mag and
  $\tau=400$~d.\label{fig:qdraws}}
\end{figure}

\begin{figure}
~[DM + EP]~
\caption{Four draws from the multi-band damped random walk model in
  the five SDSS bands.  These draws are made with $a(b)=(0.85, 0.75,
  0.65, 0.55, 0.45)$~mag in the five bandpasses, $\mu(b)=$[DM+EP], and
  $\tau=400$~d.  The model is rigid in the sense that the five bands
  are all scaled replicas of one another.\label{fig:mdraws}}
\end{figure}

\begin{figure}
~[DM + EP]~
\caption{Example of conditional
  prediction... [DM+EP]\label{fig:conditional}}
\end{figure}

\end{document}

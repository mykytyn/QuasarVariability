% This file is part of the QuasarVariability project
% Copyright 2013 David W. Hogg (NYU) and any other authors.

\documentclass[letterpaper,12pt]{article}

\newcommand{\given}{\,|\,}
\newcommand{\transpose}[1]{{#1}^{\mathsf{T}}}

\begin{document}

\section{The model}

In our language, a ``model'' is a likelihood function---a function
equal to or proportional to a pdf in data space for the data given
parameters---and a set of prior pdfs for some or all of those
parameters.  For this project, the ``data'' are a set of magnitudes


To generate the likelihood function we use here, we make
use of a Gaussian Process formulation for stochastic quasar
variability; this formulation can encompass damped random walk models
and also power-law structure functions (and indeed many other kinds of
continuous stochastic variability).  The unusual aspect of the model
we use here is that the Gaussian Process is not in any particular band
but instead in an ersatz fiducial band which can be scaled and shifted
onto the particular bands.  This permits simultaneous treatment of
multiple bands, even when the multiple bands are not observed
simultaneously.  The key idea is that the ersatz band is a latent
variable---it is never directly observed; only the scaled and shifted
versions are observed, and even these are only observed in the
presence of substantial measurement noise.

In the damped random walk model, without loss of generality (the real
photometric measurements will be scaled and shifted to match the
latent ersatz basis), the ersatz lightcurve can be described with a
zero-mean and unit-characteristic-variance Gaussian Process.  That is,
the pdf for a set of $N$ ersatz ``magnitudes'' $q_n$ that are
instantiated at a set of $N$ times $t_n$ is just
\begin{eqnarray}
p(q) &=& N(q\given 0,V)
\\
\transpose{q} &\equiv& [q_1, q_2, \cdots , q_N]
\\
V_{nn'} &=& \exp -\frac{|t_n - t_{n'}|}{\tau}
\quad ,
\end{eqnarray}
where $N(x\given\mu,V)$ is the multivariate normal for column vector
$x$ given mean vector $\mu$ and general variance tensor $V$, $q$ is
the $N$-dimensional column vector made up of all the latent ersatz
magnitudes $q_n$, $0$ is not zero but the $N$-dimensional generalization of
zero, $V$ is a $N\times N$ symmetric positive definite matrix with
elements $V_{nn'}$, and $\tau$ is the decorrelation time of the random
walk.  Some draws from this kind of Gaussian Process are shown in
\figurename~[HOGG].

\end{document}
